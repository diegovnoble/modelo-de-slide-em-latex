\documentclass[
    mode=present,
    style=dvn,
    paper=screen,
    display=slidesnotes,
    size=14pt,
]{powerdot}

\usepackage[T1]{fontenc}
\usepackage[brazilian]{babel}

\title{Teoria da Complexidade \\
    \small Notação Assintótica\\}
\author{\smallskip Prof. Diego Noble \\%
\texttt{\href{mailto:your@mail.com}{\color{pdcolor4}your@mail.com}}}%
\date{\today}

\begin{document}

\maketitle

\begin{slide}{Introdução}
    \vspace{\stretch{1}}
    \begin{itemize}
        \item Analisar algoritmos significa determinar os recursos computacionais que
        o algoritmo requer conforme o tamanho da entrada aumenta.
    \end{itemize}
    \bigskip
    
    $\rightarrow$ O objetivo desta aula é ``concretizar'' esta ideia de análise.\pause

    $\rightarrow$ Esse é passo inicial para compreender o conceito de \textit{tratabilidade}.
    \vspace{\stretch{1}}
\end{slide}

% \begin{slide}{Pior caso}
%     \vspace{\stretch{1}}
%     \begin{itemize}
%         \item 
%     \end{itemize}
%     \vspace{\stretch{1}}
% \end{slide}

\begin{slide}{Conteúdo}
    \vspace{\stretch{1}}
    \Large
    \tableofcontents[content=sections]
    \vspace{\stretch{1}}
\end{slide}
\section{Conceito de Eficiência}
\begin{slide}{Complexidade}
    \vspace{\stretch{1}}
    \Large
    \begin{itemize}
    \item \myemph{Encontrar algoritmos eficientes para solucionar problemas computacionais.}
    \end{itemize}
    Mas o que é ``\myemph{executar rapidamente}''?
    \vspace{\stretch{1}}
\end{slide}

\begin{slide}{Eficiência I}
    \vspace{\stretch{1}}
    \Large
    \begin{defn}
        Um algoritmo é eficiente se, quando implementado, executa rapidamente para instâncias reais como entrada.
    \end{defn}
    \vspace{\stretch{1}}
\end{slide}

\begin{slide}{Eficiência II}
    \vspace{\stretch{1}}
    \Large
    \begin{defn}
        Um algoritmo é eficiente se, qualitativamente e em seu pior caso, tem um desempenho superior a um algoritmo de força-bruta.
    \end{defn}
    \vspace{\stretch{1}}
\end{slide}

\begin{slide}{Eficiência III}
    \vspace{\stretch{1}}
    \Large
    \begin{defn}
        Um algoritmo é eficiente se o seu tempo de execução é polinomial.\pause
    \end{defn}
    \myemph{Mas $n^{100}$ é melhor que $n^{1+0.02(\log n)}$?}
    \vspace{\stretch{1}}
\end{slide}

\section{Notação Assintótica}

\begin{slide}{\textit{Big-oh} $\mathcal{O}$}
    \vspace{\stretch{1}}
    \begin{defn}
        \myemph{Limite assintótico superior} $T(n)$ é $O(f(n))$ \textit{se} existem constantes $c>0$ e $n_0 \geq 0$ tal que $\forall n \geq n_0$, é o caso que \myemph{$T(n) \leq c.f(n)$}.\pause
    \end{defn}
    Dizemos neste caso que $T(n)$ é limitada superiormente por $f(n)$.
    \bigskip
    \begin{tcolorbox}[title=Exemplo]
        $T(n) = 10n + 8, f(n)=n^2, c = 5, n_0 = 2$ 
    \end{tcolorbox}
    \vspace{\stretch{1}}
\end{slide}

\begin{slide}{\textit{Big-oh} $\mathcal{O}$}
    \vspace{\stretch{1}}

    \begin{figure}
        \input{./pic/big}
        \caption{$T(n){\in}\mathcal{O}(f(n))$}
    \end{figure}
    \vspace{\stretch{1}}
\end{slide}

\begin{slide}{$\Omega$}
    \vspace{\stretch{1}}
    \begin{defn}
        \myemph{Limite assintótico inferior} $T(n)$ é $\Omega(f(n))$ \textit{se} existem constantes $c>0$ e $n_0 \geq 0$ tal que $\forall n \geq n_0$, é o caso que \myemph{$T(n) \geq c.f(n)$}.\pause
    \end{defn}
    Dizemos neste caso que $T(n)$ é limitada inferiormente por $f(n)$.
    \bigskip
\vspace{\stretch{1}}
\end{slide}

\begin{slide}{$\Omega$}
    \vspace{\stretch{1}}
    \begin{figure}
        \input{./pic/omega}
        \caption{$T(n){\in}\Omega(f(n))$}
    \end{figure}
    \vspace{\stretch{1}}
\end{slide}


\begin{slide}{$\Theta$}
    \vspace{\stretch{1}}
    \begin{defn}
        \myemph{Limite assintótico estrito} $T(n)$ é $\Theta(f(n))$ \textit{se} $T(n)$ é tanto $O(f(n))$ quanto $\Omega(f(n))$.\pause
    \end{defn}
    Dizemos neste caso que $T(n)$ é limitada estritamente por $f(n)$.\pause

    \begin{itemize}
        \item Também conhecido por limite restrito.
        \item A função $T(n)$ cresce dentro de um fator constante multiplicado por $f(n)$.
    \end{itemize}
    \vspace{\stretch{1}}
\end{slide}


\begin{slide}{$\Theta$}
    \vspace{\stretch{1}}
    \begin{figure}
        \input{./pic/tight}
        \caption{$T(n){\in}\Theta(f(n))$}
    \end{figure}
    \vspace{\stretch{1}}
\end{slide}

\begin{slide}{Observações}
    \vspace{\stretch{1}}
    \begin{itemize}
        \item Dadas duas funções $g=n^2$ e $f=n + 32$, anotamos que $f(n){\in}\mathcal{O}(g(n))$ usando a seguinte notação: $f(n) = \mathcal{O}(g(n))$. 
        \item Neste caso, lemos $f(n)$ \myemph{é} $\mathcal{O}(g(n))$ ao invés de nos referirmos ao sentido de igualdade usual.
        \item Isto porque $\mathcal{O}(g(n))$ é um conjunto de funções que tem o mesmo limite assintótico superior que $g(n)$.
        \item Portanto $f(n)$ é uma função que pertence a esse conjunto.
    \end{itemize}
    \vspace{\stretch{1}}
\end{slide}

\begin{slide}[method=direct]{Pure Minted Sample}
    \vspace{\stretch{1}}
    \begin{minted}{c}
int main(void){
    return 0;
}
    \end{minted}
    \vspace{\stretch{1}}
\end{slide}

\begin{slide}[method=direct]{CodeBox Sample Minted Simple}
    \vspace{\stretch{1}}
    \begin{cbox}{Listing}
int main(void){
    return 0;
}
// cbox, cppbox, javabox, pythonbox, haskellbox.
    \end{cbox}{teste}
    \vspace{\stretch{1}}
\end{slide}

\section{Leituras}

\begin{slide}{Bibliografia consultada}
    \vspace{\stretch{1}}
    \bibliographystyle{alpha}
    \bibliography{refs.bib}
    \bibentry{Garey1979}
    \bibentry{Cormen2009}
    \vspace{\stretch{1}}
\end{slide}
\end{document}


